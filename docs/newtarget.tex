
\documentclass[11pt]{article}

\usepackage{path}

\title{How to Add a New Target}
\author{Christian Lindig}

\newcommand\PAL{{\small C-{}-}}
\newcommand\AST{{\small AST}}
\newcommand\CFG{{\small CFG}}
\newcommand\qcc{{\small QC-{}-}}
\newcommand\rtl{{\small RTL}}
\newcommand\burg{{\small BURG}}
\newcommand\cfg{{\small CFG}}
\newcommand\ocaml{{\small OCAML}}

\parindent0pt
\parskip1.5ex

% ------------------------------------------------------------------ 
\begin{document}
% ------------------------------------------------------------------ 

\maketitle

When you want to add support for a new target architecture to the {\qcc}
compiler, here is what you have to do.

\begin{enumerate}
\item Add a new target description to \path|Targets|. All targets are
      exported from here to the Lua interpreter.

\item Generate am emitter for assembly instructions in directory
      \path|gen/|.

\item Implement the assembler interface \path|Asm3.assembler| for the
      new target. The \path|Mspacasm| module might work as a template.
      There is another module \path|Sparcasm| which is currently not
      used because I don't understand it. The assembler is exported to
      the Lua interpreter in the \path|Driver2| module.

\item Implement an {\rtl} encoder for instructions and a matching
      decoder. See modules \path|Msparcenc| and \path|Msparcdec| for
      examples (source file \path|minisparc.nw|). The decoder is used by
      the assembler to decode and emit assembly code. The decoder itself
      uses the generated emitter to emit the recognized machine
      instruction.

\item Implement a code expander and export it to Lua. Module
      \path|Sparcexpander| could serve as a guide. The code expander
      uses the encoder to generate instructions.      
      
\item Extend the driver code written in Lua in file \path|main2.nw| with
      an optimizer (table \path|Opt|) and target (table \path|Targets|) for
      the new architecture.
\end{enumerate}

% ------------------------------------------------------------------ 
\end{document}
% ------------------------------------------------------------------ 
