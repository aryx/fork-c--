%
% Christian Lindig
% The Architecture of the Quick C-- Compiler
% 19 Slides
% Presented at 4th June 2002
%

\documentclass{seminar}
\usepackage{semlcmss}
\usepackage{path}
\usepackage{alltt}
% \usepackage[dvips]{pstcol}
\usepackage{graphicx}
\usepackage{semcolor}


% ----------------------------------------------------------------------------
% Slide Customization
% ----------------------------------------------------------------------------

\setlength{\slideheight}{7.5in}
\setlength{\slidewidth}{10.5in}

\renewcommand{\slideleftmargin}{2in}
%\renewcommand{\sliderightmargin}{0.2in}
\renewcommand{\slidetopmargin}{0.2in}
\renewcommand{\slidebottommargin}{0.2in}

%\definecolor{moccasin}{rgb}{1.0,0.89,0.70}
%\definecolor{darkgreen}{rgb}{0.14,0.26,0.18}
%\definecolor{lightgrey}{rgb}{0.90,0.90,0.90}
%\newrgbcolor{darkblue}{0.09766 0.09766 0.4375}
%\newrgbcolor{myblue}{.2 .36 .77}

%\newslideframe{myframe}
%    [\psset
%        { fillstyle=solid
%        , linewidth=0pt
%        , fillcolor=myblue
%        }]{\psframebox{\white #1}}    
% \slideframe{myframe}

\newrgbcolor{red}{0.95  0.27  0.30}

%\slidesmag{4}
\slideframe{none}
\slidestyle{empty}
\pagestyle{empty}
\slideframewidth1pt
\rotateheaderstrue

\renewcommand{\slidefuzz}{2cm}

% ----------------------------------------------------------------------------
% Definitions
% ----------------------------------------------------------------------------

\def\heading#1{\textsf{\textit{#1}}\vskip2pt\hrule\bigskip}
\def\subheading#1{\bigskip\textsf{\textit{#1}}\medskip}


\def\PAL{\hbox{C-{}-}}

\let\C\PAL 
\let\pal\PAL               
                        
\def\sled{{\small SLED}}
\def\sparc{{\small SPARC}}
\def\burg{{\small BURG}}
\def\asdl{{\small ASDL}}   
\def\qcc{\hbox{\small QC-{}-}}  
\def\noweb{{\small NOWEB}}
\def\ocaml{{\small OCAML}}
\def\rtl{{\small RTL}}    
\def\ir{{\small IR}}      
\def\AST{{\small AST}}   
\def\cfg{{\small CFG}}  
\def\stp{\textit{sp}}
\def\lrtl{$\lambda$-{\rtl}}

\def\x#1{{\red #1}}

\newenvironment{code}%
    {\begin{minipage}[t]{0.45\hsize}\begin{alltt}\small}
    {\end{alltt}\end{minipage}}

\newenvironment{half}%
    {\begin{minipage}[t]{0.45\hsize}}
    {\end{minipage}}



% -----------------------------------------------------------------------------
\begin{document}
% -----------------------------------------------------------------------------

\begin{slide}
    \strut\vfill
    {\Large\sf The Architecture of the Quick {\C} Compiler}
    \vspace{0.5ex}\hrule\vspace{4ex}
    {\large\textsf{Christian Lindig} 
           \texttt{<lindig@eecs.harvard.edu>}}\\[2ex]
    {\large\textsf{Harvard University}}\\[3ex]
    \vfill\vfill
\end{slide}

% ----------------------------------------------------------------------------

\begin{slide}
    \heading{The Quick {\C} Compiler}
    \begin{half}
        \begin{itemize}
        \item Compiles {\PAL} to (textual or binary) assembly code.
        \item Cross compiler with multiple back ends in one binary.
        \item Current targets: symbolic dummy target, {\small SPARC}.
        \item Back end controlled by built-in Lua interpreter.
        \item {\rtl}-based translation technique.
        \item Graph-coloring and linear-scan register allocator.
        \end{itemize}
    \end{half}
    \hfil
    \begin{half}
        \begin{itemize}
        \item Implemented in O'Caml 3.04.
        \item Extensively documented: literate program in NoWEB.
        \item Circa 100 modules, about 27\,000 lines of hand-written
              code.
        \item Machine-generated code in front and back ends: Lex, Yacc,
              {\small ASDL}, Burg, $\lambda$-{\rtl}.
        \item Implementation so far tries to avoid mutable state.
        \end{itemize}
    \end{half}
\end{slide}

% ------------------------------------------------------------------ 

\begin{slide}
    \heading{The Front End}
    \def\v{$\downarrow$}
    \let\p\path

    \begin{center}
    \begin{tabular}{c@{\quad}l}
       \path|Scan|      & records line breaks in \p|Srcmap.map|    \\
       \v               \\ 
       \p|Parse.token|  \\
       \v               \\
       \path|Parse|     & Yacc generated parser \\
       \v               \\
       \p|Ast.program|, & \p|Ast.program| generated from {\asdl} spec.\\
       \p|Srcmap.map|   & source code positions\\
       \v               & check static semantics, populate fat env.\\ 
       \path|Elab|      & evaluate constant expressions (\p|Rtleval|)\\
       \v               & propagates errors (\p|Error|)\\
       \p|Ast.program|, 
       \p|Srcmap.map|,    \\
       \p|Fenv.Dirty.env|  \\
    \end{tabular}   
    \end{center}   
\end{slide}


% ------------------------------------------------------------------ 

\begin{slide}
    \heading{The Fat Environment (\texttt{Fenv})}

    \begin{half}
    \begin{itemize}
    \item Provides meaning for names.
    \item Two flavors: \emph{clean} and \emph{dirty}.
    \item Dirty environment can contain errors.
    \item Mostly functional, but few exceptions.
    \end{itemize}
    \end{half}
    \hfil
    \begin{code}
type  kind = 
    | Proc      of Symbol.t * scope partial 
    | Code      of Symbol.t    
    | Data      of Symbol.t
    | Stack     of Rtl.exp option

type denotation =
    | Constant  of Bits.bits
    | Label     of kind
    | Import    of string option * Symbol.t
    | Register  of register
    | Continuation  of string * Symbol.t   

type ventry = (* value entry *)
    Srcmap.rgn * (denotation * Types.ty) info
    \end{code}
\end{slide}


% ------------------------------------------------------------------ 

\begin{slide}
    \heading{Error Propagation (\texttt{Error})}

    A simple value is either \path|Ok of 'a| or \path|Error|.
    
    A complex value is \path|Error|, if one of its internal values is
    \path|Error|, and \path|Ok| otherwise.

    Combinators allow to make regular functions error-aware. Heavily
    used in module \path|Elab|.

    \begin{alltt}\small
val ematch      : 'a error -> ('a -> 'b) -> 'b error
val ematchPair  : 'a error * 'b error -> ('a * 'b -> 'c) -> 'c error
module Raise :
  sig
    val option : 'a error option                -> 'a option error
    val list   : 'a error list                  -> 'a list error
    val pair   : 'a error * 'b error            -> ('a * 'b) error
  end
    \end{alltt}
\end{slide}

% ------------------------------------------------------------------ 

\begin{slide}
    \heading{Register Transfer List (\texttt{Rtl})}

    Universal notation for machine instructions: one data structure can
    describe any machine instruction.

    \begin{half}
    Four important conceptual parts:
    \begin{enumerate}
    \item Constants (boolean, bit vectors, labels).
    \item Expressions, denote values and addresses. 
    \item Locations, unify memory and registers.
    \item Operators, primitive function on values.
    \end{enumerate}
    \end{half}
    \hfil
    \begin{half}
    Module \path|Rtl| provides two views on {\rtl}s.

    Public: abstract representation, constructor functions.
        \begin{alltt}
bits  : Bits.bits -> width -> exp 
late  : string    -> width -> exp
fetch : loc       -> width -> exp
        \end{alltt}

    Private: exposed representation for pattern matching.
          \begin{alltt}
type exp = 
    | Const of const               
    | Fetch of loc * width 
    | App   of opr * exp  list
          \end{alltt}
    \end{half}
\end{slide}

% ------------------------------------------------------------------ 

\begin{slide}
    \heading{Values in the Quick {\C} Compiler (\texttt{Bits})}

    Conceptually, values are bit vectors whose interpretation is up to
    the operators using them.

    Practically, we support up to 64 bits as signed and unsigned
    integers.  Floating point value support is planned, but difficult.

    \begin{quote}
    \begin{alltt}
type t
type width = int
val width : t -> width

module S: sig  (* signed *) 
    val of_int:     int       -> width -> t 
    val of_native:  nativeint -> width -> t
    val of_string:  string    -> width -> t
    val to_int:     t -> int
    val to_native:  t -> nativeint
end
module U: sig (* unsigned *) \dots end
    \end{alltt}
    \end{quote}
\end{slide}

% ------------------------------------------------------------------ 

\begin{slide}
    \heading{Back End Phases}

    \begin{half}
    
    \begin{enumerate}
    
    \item Creation. \path|Ast3ir| translates {\AST} to
          \emph{intermediate representation} ({\small ASM}/{\cfg}).
    \item Copy in/out. \path|Copyinout| expands call nodes in {\cfg} 
          to sequence of assignments. 
          
    \item Variable Placement. \texttt{Place\-var} replaces variables 
          in {\rtl}s by temporaries. 

    \item Code Expansion. Every {\rtl} represents a machine instruction.
          Yes, but \dots

    \item Liveness Analysis. \path|Live| annotates {\cfg} nodes
          with sets of live registers.
    \end{enumerate}
    \end{half}
    \hfil
    \begin{half}
    \begin{enumerate}
    \setcounter{enumi}{5} % ugly

    \item Register Allocation. \texttt{Color\-graph} removes
          temporaries in {\rtl}s.

    \item Substitution. \path|Ast3ir| replaces \emph{late compile
          time constants} in {\rtl}s.
    \item Constant Folding. {\rtl}s are simplified by module \path|Ast3ir|.
    \item Linearization. \path|Cfg4| linearizes the {\cfg}.
    \item Recognition. A recognizer matches and emits instructions.
    \end{enumerate}
    \end{half}
    
\end{slide}

% ------------------------------------------------------------------ 

\begin{slide}
    \heading{Intermediate Representation (\texttt{Asm}/\texttt{Cfg4})}

    \begin{itemize}
    \item \path|Ast3ir| controls compilation of a translation unit (a
          file).
    \item A compilation unit contains data and procedures. Everything
          except procedures is immediately emitted using the assembler
          interface \path|Asm3|.

          \begin{alltt}
method import    : string -> Symbol.t
method align     : int -> unit
method zeroes    : int -> unit
method value     : Bits.bits -> unit
          \end{alltt}

    \item Procedures are translated into a control-flow graph (\cfg) and
          pass the phases outlined before. Finally, they are emitted as
          well:

          \begin{alltt}
method cfg_instr : Cfg4.cfg -> Symbol.t -> unit
          \end{alltt}
    \end{itemize}

    In summary: only procedures have a tangible intermediate
    representation. For the back end, a procedure is packed up as a
    \path|Proc.t| value.
\end{slide}

% ------------------------------------------------------------------ 

\begin{slide}
    \heading{Back End Procedure Representation (\texttt{Proc.t})}

\begin{quote}
\begin{alltt}\small
type t = 
    \{ symbol:   Symbol.t            (* of procedure *)
    ; cc:       Target2.cc          (* calling convention                   *)
    ; target:   Target2.t           (* target of this procedure             *)
    ; temps:    Talloc.Multiple.t   (* allocator for temporaries            *)
    ; cfg:      Cfg4.cfg            (* control-flow graph                   *)
    ; incoming: Block.t             (* stack - incoming area                *)
    ; outgoing: Block.t             (* stack - outgoing area                *)
    ; stackd:   Block.t             (* stack - user stack data              *)
    ; conts:    Block.t             (* pairs of pointers for conts *)
    ; priv:     Automaton2.t        (* stack - spill slots etc - still open *)
    ; eqns:     Const2.t list       (* eqns for compile time consts *)
    \}
\end{alltt}
\end{quote}
\end{slide}


% ------------------------------------------------------------------ 

\begin{slide}
    \heading{Control Flow Graph (\texttt{Cfg4})}

    Together with {\rtl}s, our most important data structure.
    
    \begin{half}
    \begin{itemize}
    \item Imperative: nodes are objects, edges are pointers.
    \item Every node carries an {\rtl}.
    \item Internal implementation is object-oriented.
    \item {\cfg} is built bottom up by \path|Ast3ir|.
    \item Functions for mutation and traversal.
    \end{itemize}
    \end{half}
    \hfil
    \begin{code}
mk: target:Target2.t -> nop:Rtl.rtl -> cfg
entry:      cfg -> node   
exit:       cfg -> node
gm_assign:  cfg -> Rtl.rtl -> succ:node 
                           -> node
gm_return:  cfg -> Rtl.rtl -> int * int 
                           -> node
gm_goto:    cfg -> Rtl.rtl -> node list 
                           -> node
next:       node -> node option
instr:      node -> Rtl.rtl
    \end{code}
\end{slide}

% ------------------------------------------------------------------ 

\begin{slide}
    \heading{Stack Frame (\texttt{Block})}

    A stack frame is an algebraic composition of memory blocks,
    represented by \path|Block.t|.
    
    \begin{quote}
    \begin{alltt}
type t
type placement = High | Low
val mk         : base:Rtl.exp -> size:int -> alignment:int -> t
val cat        : t -> t -> t
val overlap    : placement -> t -> t -> t
    \end{alltt}
    \end{quote}
    
    Sources of blocks: stack data, initialized data, global variables, 
    slot allocation, calling conventions.

    Most blocks are allocated piecewise using an \path|Automaton2.t|
    value that encapsulates the allocation policy.
\end{slide}

% ------------------------------------------------------------------ 

\begin{slide}
    \heading{Automata for Resource Allocation (\texttt{Automaton2})}

    Abstraction for a pool of abstract \emph{locations} that can be
    requested.  

    Locations can be complex, like two registers, or a register and a
    memory cell.

    \begin{quote}
    \begin{alltt}
type t
type loc    
    
val mk       : spec -> address:Rtl.exp -> t
val allocate : t -> width:int -> hint:(string option) -> loc
val fetch    : loc -> width:int -> Rtl.exp
val store    : loc -> Rtl.exp -> width:int -> Rtl.rtl
val freeze   : t -> Block.t
    \end{alltt}
    \end{quote}

    Important applications: slot allocation in stack frame,
    implementation of calling conventions. Both are part of a target
    description (\path|Target2.t|).
\end{slide}

% ------------------------------------------------------------------ 

\begin{slide}
    \heading{Target (\texttt{Target2})}

    Complex record to describe all aspects of a target. Important
    entries:

    \begin{quote}\small
    \begin{alltt}
spaces:  Space.t list;      (* memory, register, temporaries *)
spill :  (Rtl.space -> Space.t) -> Register.t -> Rtl.loc -> Rtl.rtl list;
reload:  (Rtl.space -> Space.t) -> Register.t -> Rtl.loc -> Rtl.rtl list;
cc    :  string -> cc;      (* calling convention *)
stack_slots:    Automaton2.spec;

type cc =      
    \{ sp:           Rtl.loc          (* stack pointer                      *)
    ; return:       Rtl.rtl          (* machine instr passed to Cfg.return *)
    ; proc:         Automaton2.spec  (* pass parameter to a procedure      *)
    ; cont:         Automaton2.spec  (* pass parameter to a continuation   *)
    ; ret:          Automaton2.spec  (* return values                      *)
    ; allocatable:  Register.t list  (* regs for reg-allocation            *)
    \}
\end{alltt}
    \end{quote}
    
\end{slide}

% ------------------------------------------------------------------ 

\begin{slide}
    \heading{Better Living through Lua (\texttt{Main2}/\texttt{Backplane}/\texttt{Lua})}

    \begin{code}
function Main.Util.cc (target,file) 
    local ast = Driver.parse(file) 
    local asm = target.asm(Driver.stdout) 
    local env = Driver.check(ast,asm) 
    local asm = 
        Driver.compile
        (target.opt, ast, target.id, env)
    Driver.assemble(asm)
end    

function Opt.sparc(proc)
    local action = seq
        \{ B.single(Expand.stages.placeVars)
        , B.single(Expand.stages.sparc)
        , Default.allocator
        \}
    B.run(action, proc, \{\})
end
    \end{code}
    \hfil
    \begin{half}
    \begin{itemize}
    \item Lua 2.5 implementation in O'Caml.
    \item Extensions distributed accross modules.
    \item At startup, \path|qc--| executes \verb|qc--.lua| that
          controls the compiler. 
    \item All important phases are accessible from Lua.
    \item Command line arguments are handled with Lua code.
    \item Interactive sessions with the Lua interpreter.
    \end{itemize}
    \end{half}
\end{slide}    

% ------------------------------------------------------------------ 

\begin{slide}
    \heading{Register Allocation (\texttt{Colorgraph})}
    
    Two register allocators: \emph{linear scan} and
    \emph{graph-coloring} register allocator. 

    The graph-coloring allocator is split up into phases that are
    exported to Lua and can be configured using the Lua startup code.
    \begin{quote}\small
    \begin{alltt}
CG.stages =
    \{ liveness          = B.mk("liveness",          CG.liveness)
    , build             = B.mk("build",             CG.build)
    , setColors         = B.mk("setColors",         CG.setColors)
    , makeWorklist      = B.mk("makeWorklist",      CG.makeWorklist)
    , simplify          = B.mk("simplify",          CG.simplify)
    , coalesce          = B.mk("coalesce",          CG.coalesce)
    , freeze            = B.mk("freeze",            CG.freeze)
    , selectSpill       = B.mk("selectSpill",       CG.selectSpill)
    , assignColors      = B.mk("assignColors",      CG.assignColors)
    , haveSpilledTemps  = B.mk("haveSpilledTemps",  CG.haveSpilledTemps)
    , \dots
    , applyColors       = B.mk("applyColors",       CG.applyColors)
    , printCG           = B.mk("printCG",           CG.printCG)
    , printCFGLive      = B.mk("printCFGLive",      CG.printCFGLive)
    , pCFG              = B.mk("pCFG",              CG.pCFG)
    \}
    \end{alltt}
    \end{quote}
\end{slide}    

% ------------------------------------------------------------------ 

\begin{slide}
    \heading{Code Expander (\texttt{Expander}, \texttt{Sparcexpander})}

    An expander is target specific; it rewrites every {\rtl} in the
    {\cfg} such that each resulting {\rtl} represents a machine
    instruction. 

    \path|Expander| is manually written, \path|Sparcexpander| and
    \path|Dummyexpander| are generated from Burg rules.

    \begin{quote}\small
    \begin{alltt}
addr:       App2("add", reg, rc)
            \{: 
                reg >>= fun r  ->
                rc  >>= fun rc ->
                return (I.generala r rc)
            :\}

addr:       reg         \{: reg >>= fun r -> return (I.indirecta r) :\}
addr:       const13     \{: return (I.absolutea const13) :\}

mem:        Cell('m', agg, width, addr, ass) \{: addr :\}

stmt:       Store(mem, reg, width) [1]       
            \{: mem >>= fun m -> reg >>= fun r -> exec (I.st r m) :\}    
    \end{alltt}
    \end{quote}
\end{slide}

% ------------------------------------------------------------------ 

\begin{slide}
    \heading{Code Constructors and Recognizer (\texttt{Msparc*})}

    Currently manually modified machine-generated code.
    
    Encoder:
    
    \begin{quote}\small
    \begin{alltt}
let ld address rd = 
    Rtl.store rd 
        (Rtl.fetch (Rtl.cell Rtl.none 'm' Rtl.BigEndian 32 address) 32) 32
    \end{alltt}
    \end{quote}

    Decoder and Emitter:
    
    \begin{quote}\small
    \begin{alltt}
| RP.Rtl [(RP.Const (RP.Bool true), RP.Store (RP.Cell ('r', Rtl.Identity, 
    32, RP.Const (RP.Bits rd), _), RP.Fetch (RP.Cell ('m', 
      Rtl.BigEndian, 32, RP.App (("add", [32]), [RP.Fetch (RP.Cell ('r', 
          Rtl.Identity, 32, RP.Const (RP.Bits rs1), _), 32); 
          RP.Const (RP.Bits arg13)]), _), 32), 32))] 
when Base.to_bool (Bitops.fits_signed arg13 13) -> 
   Instruction.ld (Instruction.generala (Bits.U.to_native rs1) 
         (Instruction.imode (Bits.U.to_native arg13))) 
     (Bits.U.to_native rd) 
    \end{alltt}
    \end{quote}
\end{slide}

% ------------------------------------------------------------------ 

\begin{slide}
    \heading{Source Code Organization}

    \begin{minipage}[c]{0.45\hsize}\begin{alltt}\small
CVS/        config/    mwb/
INSTALL     doc/       ocaml-3.02.patch
LAUNDRY     examples/  ocaml-bug/
README      figures/   rtl/
asdl/       gen/       specialized/
aug99/      lib/       src/
bin/        lua/       tdpe/
camlburg/   man/       test/
ccl-suite/  mk/        tmp/
cllib/      mkfile     tools/
   \end{alltt}\end{minipage} 
    \hfil\vrule\hfil
    \begin{half}
        \begin{tabular}{ll}
        \path|src/|     & main compiler source \\
        \path|lua/|     & Lua interpreter source \\
        \path|rtl/|     & {\rtl} declaration \\
        \path|gen/|     & {\lrtl}-generated files: \\
                        & emitter, constructors, \\
                        & recognizer \\
        \path|doc/|     & musings about problems \\
        \path|cllib/|   & generic modules: pretty \\
                        & printer, parser combinators \\
        \path|config/|  & shared mkfile rules, \\
                        & \LaTeX~macros\\
    \end{tabular}
    \end{half}

    No need to remember the details: \path|mk| does it for you.
\end{slide}

% ------------------------------------------------------------------ 


% -----------------------------------------------------------------------------
\end{document}
% -----------------------------------------------------------------------------

