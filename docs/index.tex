\documentclass{article}% ===> this file was generated automatically by noweave --- better not edit it
%
% $Id: index.nw,v 1.4 2006-04-10 21:28:53 nr Exp $
%
% Use 'mk index.dvi' to build a printable file.
\usepackage{noweb}

%
% Macros we use over and over again.
%

\ifx\PAL\undefined
    \def\PAL{C-{}-}
    \newcommand\module[1]{\texttt{#1.nw}} 

    \let\C\PAL 
    \let\pal\PAL               
                            
    \def\mips{{\small MIPS}}
    \def\sled{{\small SLED}}
    \def\sparc{{\small SPARC}}
    \def\burg{{\small BURG}}
    \def\asdl{{\small ASDL}}   
    \def\qcc{{\small QC-{}-}}   
    \def\noweb{{\small NOWEB}}
    \def\ocaml{{\small OCAML}}
    \def\rtl{{\small RTL}}    
    \def\ir{{\small IR}}      
    \def\AST{{\small AST}}   
    \def\cfg{{\small CFG}}  
    \def\sp{\textit{sp}}
    \def\vfp{\relax\ifmmode\mathit{VFP}\else {\small VFP}\fi}
    \def\sp{\relax\ifmmode\mathit{SP}\else {\small SP}\fi}
    \def\gets{\mathrel{:=}}
    \def\remark#1{\marginpar{\raggedright\hbadness=10000
            \def\baselinestretch{0.8}\scriptsize
            \it #1\par}}

    \def\interface#1{\section{Interface: #1}}
    \def\implementation#1{\section{Implementation: #1}}
    \newenvironment{cl}{\begin{quote}\it}{\end{quote}}
\fi


% l2h substitution mips MIPS
% l2h substitution sled SLED
% l2h substitution burg BURG
% l2h substitution sparc SPARC
% l2h substitution cfg CFG
% l2h substitution dots ...
% l2h substitution AST AST
% l2h substitution ast AST
% l2h substitution rtl RTL
% l2h substitution RTL RTL
% l2h substitution ir IR
% l2h substitution noweb NOWEB
% l2h substitution qcc QC--
% l2h substitution C C--
% l2h substitution PAL C--
% l2h substitution pal C--
% l2h substitution asdl ASDL
% l2h substitution ocaml OCaml
% l2h substitution forall forall
% l2h macro module 1 <a href="#$1.html"><tt>#1.nw</tt></a>
% l2h substitution gets <-
% l2h substitution mapsto :->
% l2h substitution prec @<<<
% l2h substitution sum sum
% l2h substitution to ->
% l2h substitution sp sp

% l2h ignore RequirePackage {
% l2h ignore newtoks {

%%%%%%%%%%%%%%%%%%%%%%%%%%%%%%%%%%%%%%%%%%%%%%%%%%%%%%%%%%%%%%%%
%
% support for noweb boxes

\makeatletter
% nwboxes --- support for boxed types in ML codes

%% \ProvidesPackage{nwboxes}
\RequirePackage{noweb}

\let\nwboxes@ostartdef=\nwstartdeflinemarkup
\let\nwboxes@oenddef=\nwenddeflinemarkup

\newtoks\@nwtyboxes
\newif\if@nwboxespending
\@nwboxespendingfalse

\newcommand\@nwboxesnospecials{\@makeother\'\@makeother\_\@makeother\$\obeyspaces}

\newcommand\nwaddbox{%
  \begingroup\@nwboxesnospecials\@nwaddbox
}
\newcommand\@nwaddbox[1]{%
  \global\@nwboxespendingtrue
  \global\@nwtyboxes=\expandafter{\the\@nwtyboxes#1\\}%
  \endgroup
  \ignorespaces
}

% l2h ignore @nwdeeperboxspace =
% l2h ignore @nwtyboxes =



\newdimen\@nwdeeperboxspace
\newcommand\deeperbox[1]
  {\@nwdeeperboxspace=#1\relax\advance\@nwdeeperboxspace by 5.5pt}
\global\@nwdeeperboxspace=0pt

%% original -- boxes to left of markup

\def\nwstartdeflinemarkup{\nwboxes@ostartdef
  \if@nwboxespending
    {\nw@boxfontsize
     \penalty500\hskip -1em\mbox{}\nobreak\hfill\nobreak
     \@nwboxesnospecials\nobreak
     \setlength\fboxsep{1pt}%
     \smash{\fbox{\begin{tabular}[t]{@{\,}l@{\,}}\the\@nwtyboxes\end{tabular}}}}%
     \nobreak\vrule width0pt height10.5pt depth\@nwdeeperboxspace
     \@nwdeeperboxspace=0pt
  \fi
  \global\@nwboxespendingfalse
  \global\@nwtyboxes={}%
}
\def\nwenddeflinemarkup{\nobreak\hskip \nwdefspace minus \nwdefspace\nobreak}
\def\nw@boxfontsize{\small}
\def\nwboxfontsize#1{\def\nw@boxfontsize{#1}}

%% new: boxes below markup

\let\nwstartdeflinemarkup=\nwboxes@ostartdef
\def\nwenddeflinemarkup{%
  \nobreak
  \if@nwboxespending
    {\small
     \@nwboxesnospecials\nobreak
     \setlength\fboxsep{1pt}%
     \smash{\llap{%
       \begin{tabular}[t]{@{}l@{}}\mbox{}\\
        \fbox{\begin{tabular}[t]{@{\,}l@{\,}}\the\@nwtyboxes\end{tabular}}%
       \end{tabular}}}}%
%\showthe\@nwdeeperboxspace
     \nobreak\vrule width0pt height8.5pt depth \@nwdeeperboxspace
     \nobreak
  \fi
  \hskip\nwdefspace\nobreak
  \global\@nwboxespendingfalse
  \global\@nwdeeperboxspace=0pt
  \global\@nwtyboxes={}%
}

\makeatother

%       end support for noweb boxes
%
%%%%%%%%%%%%%%%%%%%%%%%%%%%%%%%%%%%%%%%%%%%%%%%%%%%%%%%%%%%%%%%%


\begin{document}

% ------------------------------------------------------------------ 
\section{The Module Landscape of the Quick C\hbox{-{}-} Compiler}
% ------------------------------------------------------------------ 

\ifx\module\undefined
  \newcommand\module[1]
    {\hyperlink{module.#1}{\texttt{#1} module} (Ch\ref{#1-module}, p\pageref{#1-module})}
  \newcommand\inmodule[2]{\hyperlink{module.#1}{\emph{#2}} (\texttt{#1.nw})}
  \newcommand\markmodule[1]{\hypertarget{module.#1}{\label{#1-module}}}
  \renewcommand\module[1]{\texttt{#1.nw}}
\fi

% l2h macro module 1 <a href="#$1.html"><tt>#1.nw</tt></a>
% l2h macro inmodule 2 <a href="#$1.html">#2</a>
% l2h ignore markmodule {

% l2h let htmlcontents tableofcontents

% l2h ignore catcode ==

\begin{center}
\textbf{Last major revision: June 2002}\\
\catcode`\$=10 % space
Last CVS checkin: $Date: 2006-04-10 21:28:53 $
\end{center}


The Quick {\PAL} compiler currently is a collection of over one hundred
modules. This document tries to sketch the landscape and to identify
some landmarks for orientation.

% ------------------------------------------------------------------ 
\subsection{Driver}
% ------------------------------------------------------------------ 

The Quick {\PAL} compiler is held together by a few driver modules that
interact with the user. 

\begin{itemize} 

\item \module{../src/main2}. The first function executed in the compiler
is {\Tt{}Main.main\nwendquote}. It reads command line arguments, initializes the
built-in Lua interpreter, catches exceptions, and defines the return
code. 

All important phases in the compiler have an interface to the Lua
interpreter. The Lua code that controls the flow of data between phases
is defined in module {\Tt{}Main\nwendquote}. After the interpreter has been booted
with this code, function {\Tt{}Main.main\nwendquote} passes the command line arguments
into the interpreter for evaluation. From there on the Lua code controls
the compiler.

\item \module{../src/driver2}. This module exports front end data
structures and functions to the Lua interpreter. For example,
{\Tt{}Driver2\nwendquote} implements a new Lua data type {\Tt{}ast\nwendquote} that represents the
abstract syntax of a {\PAL} source file. 
\end{itemize}

% ------------------------------------------------------------------ 
\subsection{Front End}
% ------------------------------------------------------------------ 

The \emph{front end} reads a {\PAL} source file and translates it into a
target-independent intermediate representation which the \emph{back end}
will later translate to machine code. Almost all aspects of the front
end are independent independent of the actual target. All syntactic and
most semantic errors in a {\PAL} program are detected in the front end.
It is the main source of error messages for the user.

\begin{itemize}
\item \module{../src/parser}. This module defines the scanner and
parser for the {\PAL} language. The scanner is derived from an OCamlLex
specification, the parser from an OCamlYacc specification. The main
result from scanning and parsing is the abstract syntax tree of the
source file.

\item \module{../gen/ast}. The abstract syntax data type is derived
from an {\asdl} specification.

\item \module{../cllib/srcmap}. An OCamlYacc-generated parser represents
a source code position in a file as a character offset from the
beginning of the file. This representation makes it difficult to
annotate error messages with comprehensible source code locations. The
{\Tt{}Srcmap\nwendquote} module provides a translation for offset-based source code
positions to file- and line-oriented positions. Such a map is maintained
by the scanner and returned by the parser together with the abstract
syntax tree of a source file.

\item \module{../src/fenv}. The fat environment provides the symbol
table of the compiler. This data structure records the denotation of
names in the {\PAL} source file that is currently translated.

\item \module{../src/elab}. This module checks the static semantics of a
source file, reports errors, and builds up the fat environment. The
module does not stop after the first error but uses error propagation
combinators (\module{../src/error}) to find as many errors as possible,
even in the presence of errors. 

{\PAL} names can denote compile-time constants and type aliases. The
corresponding compile-time expressions and type aliases are resolved by
module {\Tt{}Elab\nwendquote} and the results are recorded in the fat environment.

\item \module{../src/types}. The static semantics of \C~require type
checks:  types of arguments to primitive functions must match the types
of their formal parameters,and {\Tt{}if\nwendquote}-statements are governed by an
expression of type {\Tt{}bool\nwendquote}.  This module lays out the type language for
this purpose.
 
\item \module{../src/expcheck}. The {\qcc} compiler type checks {\AST}
expressions during several phases: evaluation of constant expressions,
static semantics, translation to intermediate representation.  To avoid
code duplication, type checks for expressions are factored out into this
module.

\item \module{../src/error}. The idea of error-handling in {\qcc} is not
to stop in all cases after an error but to mark results as either good
or bad.  Bad values can be used for further computation although the
overall result will be bad.  The hope is, to detect even more errors
that are \textit{not} caused by any previous error. The
error-propagating combinators in this module allow to make ordinary
functions error-aware, allowing them to deal with inputs that might be
bad.

\end{itemize}

New front end modules as of January 2004:
\begin{itemize}
\item
\module{../src/nast}.
``Normalized AST.''
The Quick~{\PAL} AST follows the concrete syntax directly.
As such, it is very flexible for the front end and correspondingly
complicated to process.
The goal of this module is to put the AST into a sort of a normal
form.
The normal form collects and segregates declarations for simpler
processing.
Every AST is normalizable, so this module never causes an error.
\item
\module{../src/nelab},
\module{../src/elabexp},
\module{../src/elabstmt}.
Elaboration of an AST.
These modules split what was done by \module{../src/elab} in the old
front end.
Unlike the old front end, they not only detect errors but also put the
program into an elaborated normal form.
Thus, some work that was deferred to \module{../src/ast2ir} in the old
front end is done here in the new front end.

To ease understanding, elaboration is split across three modules:
\module{../src/elabexp} for expressions,
\module{../src/elabstmt} for statements,
and 
\module{../src/nelab} for programs.

These modules are the \emph{only} modules that detect and return error values.
Upstream, no errors are detected, and downstream, all errors are
assumed to have been detected here.
\end{itemize}



% ------------------------------------------------------------------ 
\subsection{Translation to Intermediate Representation}
% ------------------------------------------------------------------ 

Front end and back end are connected by the \emph{intermediate
representation}. During the translation, that last step of
the front end is to create the intermediate representation of the
program under translation.

\begin{itemize} 

\item \module{../src/ast3ir}.  The {\Tt{}translate\nwendquote}
function takes an environment, the actual target and a program and
translates it into an abstract assembler program.  The assembler is
functorized over the data type of instructions. This module puts a
procedure body into a single instruction by basically treating a
control-flow graph as an instruction. 

\item \module{../src/copyinout}. The details of how parameters are
passed to a procedure and results are returned are covered by a calling
convention. The implementation of a calling conventions is a sequence of
load and store instructions that move parameters into the registers and
memory slots. The {\Tt{}Copyinout\nwendquote} module creates such sequences of
assignments nodes in the {\cfg} for parameter passing. It is used during
the creation of the intermediate representation (by
\module{../src/ast3ir}).

\end{itemize}


% ------------------------------------------------------------------ 
\subsection{Intermediate Representation}
% ------------------------------------------------------------------ 

A program has two principal parts: initialized and uninitialized data
like strings, buffers, and constants, and code. In the intermediate
representation only code has is held in a data structure. Data is
directly emitted to an assembler during the translation to the
intermediate representation by module \module{../src/ast3ir}.

\begin{itemize}

\item \module{../src/cfg4}.  The executable part of a procedure is
represented as a control-flow graph (\cfg). The graph is polymorphic
over instructions that are embedded into its nodes. The most important
specialization is when {\rtl}s are used as instruction representation.

\item \module{../src/asm3}.  The \module{../src/astr3ir} module uses
this interface to access text-based and binary assemblers. Assembler
aspects like symbols and relocatable addresses (from module
\module{../rtl/symbol}) escape the assembler and become part of symbol
tables and expressions.  To avoid that all aspects of the compiler must
be functorized over the different assemblers we have chosen to use a
very polymorphic interface that uses Objective Caml's object system.

\item \module{../src/cfgprop}. We may attach arbitrary properties to
each node of a control-flow graph.  Because the set of properties is
expected to evolve while the flow-graph interface remains stable, we put
the properties in a separate module.

\item \module{../src/cfgutil}. The \module{../src/cfg4} module provides
the essential functionality for control-flow graphs ({\cfg}s). In order
to separate essential functionality from nice-to-have functionality this
module provides additional functions on {\cfg}s whose implementation do
not require knowledge of the internal details of \module{../src/cfg4}.

\end{itemize}

% ------------------------------------------------------------------ 
\subsection{Assemblers}
% ------------------------------------------------------------------ 

An assembler is an implementation of the \module{../src/asm3} interface. 

\begin{itemize} 

\item \module{../src/asdlasm.nw}. This module implements
an assembler that emits the {\rtl}s of a procedure in {\asdl} format.
Other assembler instruction do not create output.

\item \module{../src/astasm}. This is an assembler that emits {\PAL}. It
implements the {\Tt{}Asm3.assembler\nwendquote} interface for assemblers in the {\qcc}
compiler.

\item \module{../src/dotasm}. This module implements an assembler that
emits every procedure as a graph in \textsc{dot} format. The assember
constructor receives the output channel for the assembler.

\item \module{../src/msparcasm}. This module provides an assembler that
emits directives and instructions in {\sparc} assembly syntax. The
interface conforms to the {\Tt{}Asm3\nwendquote} assembler interface.

\item \module{../src/sparcasm}. This is an assembler that emits SPARC
assembly language.  It implements the {\Tt{}Asm3.assembler\nwendquote} interface for
assemblers in the {\qcc} compiler.  It should be functorized to deal
with differences between SunOS and Solaris, but to hell with SunOS.
Unused.

\end{itemize}

% ------------------------------------------------------------------ 
\subsection{Stack Frame and Calling Conventions}
% ------------------------------------------------------------------ 

A procedure keeps local data at run-time in registers and in its stack
frame. It is the compiler's responsibility to organize a the stack
frame. The modules below all deal with memory blocks that are ultimately
composed into a stack frame.

\begin{itemize}

\item \module{../src/block}. A {\Tt{}Block.t\nwendquote} value represents a memory
block. Memory blocks can be composed to larger blocks. This abstraction
is used to model slots in a stack frame, parameter passing areas in a
frame, entire stack frames, and memory for data.

\item \module{../src/memalloc}.  This module provides an applicate
abstraction to allocate a memory block incrementally by a sequence of
{\Tt{}allocate\nwendquote} and {\Tt{}align\nwendquote} directives.  The abstraction is similar to a
pointer that is advanced and aligned to reserver memory. At any point
the value of the  pointer can be used as an address. After allocation
is complete, the {\Tt{}t\nwendquote} value is {\Tt{}frozen\nwendquote} and the allocated block is
returned. 

\item \module{../src/automaton2}. Allocation of registers for parameter
passing and slot allocation can be hidden behind an automaton. An
automaton knows a set of locations and hands them out in response to
queries. A location can be a hardware register, a memory location, or a
combination of both. The details are hidden behind a common abstraction.
Automata are target specific and implement allocation policies.

\item \module{../src/contn} A continuation is a pinter-sized {\PAL}
\emph{value}. As such, it can be passed around and stored in registers.
However, not all data necessary to represent a continuation fit into a
pointer and therefore the pointer points to a pair of two values in
memory: first, a pointer to the code belonging to a continuation, and
second, a stack pointer value. We call this pair the continuation
\emph{representation}. The representation is stored in the activation
record of a procedure and is initialized when the procedure becomes
active.  To deal with the different aspects of a continuation, this
module provides an abstraction.

\item \module{../src/eqn}, \module{../src/const2}. The memory block
abstraction  represents constraints by equations over {\rtl}
expressions. 

The {\Tt{}Const2\nwendquote} module allows to build an equation system by equating
{\rtl} expressions.  The equation system determines the value of late
compile-time constants that are part of the participating expressions.
For solving the equation system, equations are transformed into linear
equations over integers.

{\Tt{}Eqn\nwendquote} provides a solver for linear equations: the solver takes a set
of linear equations as input and finds the values of the embedded
variables. A linear equation has the form $\sum_{i} c_i t_i = 0$ where
each coefficient $c_i$ is an integer and each term $t_i$ is either a
variable ($v$) or a non-zero constant ($k$): $t ::= v | k$.

\end{itemize}


% ------------------------------------------------------------------ 
\subsection{Register Transfer Lists}
% ------------------------------------------------------------------ 

The register transfer list (\rtl) is a central data structure in the
compiler. A register transfer list is a general data structure that can
represent any machine instruction. Register-transfer lists capture the
meaning of effects, like storing values into memory or register
location.  They are used as part of the abstract representation of
{\PAL}.  The meaning of a register-transfer list is well defined and
especially independent from any properties of a target architecture.
{\rtl}s are carried in nodes of the control-flow graph and created by
the \module{ast3ir} module.


\begin{itemize}

\item \module{../rtl/rtl}. The definition of a {\rtl}. The interface is
divided into three parts:  \emph{public}, \emph{private}, and
\emph{common}.  The public part provides constructor functions to build
{\rtl}s but hides their actual implementation.  The private part reveals
these and is intended only for parts of the back end that does
re-writing of {\rtl}s.  Both private and public parts share
\emph{common} types.  The public part includes function that converts
public values into private values and thus makes their details
accessible. 

\item \module{../src/space}. The {\rtl} framework groups memory cells
(main memory, registers, temporaries) into named spaces. The cells
forming a space share properties but the {\rtl} framework does not
announce them. This module provides a type to describe an {\rtl} space. 


\item \module{../src/rtlutil}. Functions to observe and transform
{\rtl}s. For example, function to observe the width of a location or an
expression, substitution routines, and functions to create a string
representation of an {\rtl}.

\item \module{../src/rtlop}. This module provides the translation of a
{\PAL} operator to an {\rtl} operator.  {\PAL} and {\rtl} operators are
closely related: a {\PAL} operator is in general an instance of a
polymorphic {\rtl} operator.

\item \module{../src/rtleval2}. An {\rtl} expression can denote a
compile-time, link-time, or run-time value. The goal is to evaluate any
{\rtl} expression as early as possible to minimize computation at
run-time. This module provides routines to evaluate {\rtl} expressions
and to simplify {\rtl}s by evaluating expressions inside of them. The
promise is, that the returned {\rtl}s and expressions are simpler as the
ones provided.

\item \module{../src/rtldebug}. Functions to type check an {\rtl},
i.e.~to check its internal consistency. Used for debugging.

\end{itemize}

% ------------------------------------------------------------------ 
\subsection{Values (Bit Vectors)}
% ------------------------------------------------------------------ 

{\PAL} treats all values as bit vectors. Any interpretation of a value
as an integer or floating point value is left to operators working on
values.

\begin{itemize}

\item \module{../rtl/bits}.  This module provides an abstract bit vector
type {\Tt{}bits\nwendquote}. It is used for holding {\PAL} values.  Bit values can be
converted to and from numbers and strings. Conversion implies an
\emph{interpretation} of a value. Most commonly, values can be
interpreted as signed or unsigned integers.  Therefore, constructors and
observers of {\Tt{}bits\nwendquote} values are organized along these lines. The
conversions try hard not to assume a certain representation for {\ocaml}
integers.

\item \module{../rtl/bitops}. This module encapsulates operations on bit
vectors.  Its primary use is to support constant evaluation, but it
is also used to evaluate `early' expressions during {\rtl} creation.
Another mayor client is the encoding and decoding of instructions in
{\rtl} representation.

\item \module{../gen/rtlasdl}, \module{../src/rtlx}. To allow external
software to read {\rtl}s we have generated, we export them using
{\asdl}.  Since the {\rtl} data type is defined as a {\ocaml} data type
and not as a {\asdl} data type, we create an additional definition of an
{\rtl} using the {\asdl} specification language. We then convert an
existing {\rtl} into an \asdl-based {\rtl} which we can export.
 
\end{itemize}

% ------------------------------------------------------------------ 
\subsection{Registers, Liveness, and Temporaries}
% ------------------------------------------------------------------ 

A register-like location is an {\rtl} location whose address is always
an compile-time integer constant. Because registers are central to the
compiler, they have their own data structure and functions.

\begin{itemize}

\item \module{../src/register}. A {\Tt{}Register.t\nwendquote} describes a
target-specific register-like location.  Such a location includes
hardware-registers or temporaries, but not spill locations.  The
distinguishing feature of a register is a fixed address within its
space: the address is an integer rather than a general expression.

\item \module{../src/live}. Liveness analysis determines for every node
in a flow graph (see module \module{../src/cfg4}) the set of registers
that will be used in that node's successors. 


\item \module{../src/lifetime}. A lifetime is a set of disjoint
intervals.  In the context of the {\qcc} compiler a lifetime represents
the parts in a linear program representation a variable is live.  The
fact that a lifetime is a set of disjoint intervals means, that a
variable can be live in different parts of a (linearized) program that
are not adjacent to each other.

\item \module{../src/talloc}. Multiple parts of the compiler may need to
allocate temporaries.  Here we provide simple allocators for
temporaries.

\end{itemize}

% ------------------------------------------------------------------ 
\subsection{Variable Placement}
% ------------------------------------------------------------------ 

Initially, {\rtl}s as part of {\cfg} nodes include variables. They must
be replaced by temporaries.

\begin{itemize}
\item \module{../src/placevar}. A toy variable placer.
\end{itemize}

% ------------------------------------------------------------------ 
\subsection{Register Allocation}
% ------------------------------------------------------------------ 

Register allocation replaces temporaries in {\cfg} nodes by hardware
registers. When necessary, registers are spilled to stack slots.

\begin{itemize}

\item \module{../src/ocolorgraph}. A graph-coloring register allocator
with whose individual components can be controlled from Lua.

\item \module{../src/linscan}. A linear-scan register allocator with a
minimal Lua interface.

\end{itemize}

% ------------------------------------------------------------------ 
\subsection{Lua Interpreter}
% ------------------------------------------------------------------ 

The compiler is controlled by a built-in Lua 2.5 interpreter. All
important phases of the compiler have interfaces to Lua which allows
them to be controlled by Lua code. The implementation of the Lua
interpreter is in its own sub directory {\Tt{}lua/\nwendquote}.

\begin{itemize}

\item \module{../lua/lua}. Central module that provides access to the
Lua interpreter.

\item \module{../src/luautil}. This module provides a new Lua primitive
that allows to load a Lua file that is searched along a search path.

\end{itemize}

% ------------------------------------------------------------------ 
\subsection{Back Plane}
% ------------------------------------------------------------------ 

The backplane is a set of new Lua primitives that allows to control
compiler phases with Lua interface from Lua.

\begin{itemize}

\item \module{../src/backplane}. The backplane allows the user to
customize the backend of the compiler. The building blocks of the
backplane are the stages between which the user can define a control
flow. Stages represent compiler components, such as the optimizer or the
register allocator. The backplane can also be used with a finer level of
granularity of compiler components, such as individual optimization
stages.

\end{itemize}

% ------------------------------------------------------------------ 
\subsection{Procedure Representation in the Back End}
% ------------------------------------------------------------------ 

The compiler's back end is mostly concerned with the translation of
procedures. A procedure in the back end is represented by a record that
groups all associated informations.

\begin{itemize}
    \item \module{../src/proc}. Data type definition for procedure
    representation.
\end{itemize}


% ------------------------------------------------------------------ 
\subsection{Targets}
% ------------------------------------------------------------------ 

The compiler has to know the properties of the target it is compiling
code for. Here are the modules that provide the data structures and
values for this information.

\begin{itemize}
\item \module{../src/target2}. Data structure {\Tt{}Target2.t\nwendquote} that
describes a target,

\item \module{../src/targets}. All targets the compiler knows about.
Defined a number of {\Tt{}Target2.t\nwendquote} values. Targets are exported to Lua.

\end{itemize}

% ------------------------------------------------------------------ 
\subsection{Code Expansion}
% ------------------------------------------------------------------ 

A code expander expands nodes in the {\cfg} such that each node's {\rtl}
represents a single machine instruction. Code expansion is target
specific and handled by different modules. Each module has a Lua
interface and thus can be selected at run-time.

\begin{itemize} \item \module{../src/expander}. A toy code expander from
the early days of the compiler.

\item \module{../src/dummyexpander}. The {\rtl}s generated by the
\module{../src/ast3ir} module violate the machine invariant that every
{\rtl} must be representable as a machine instruction. A code-expander
establishes this invariant by simplifying {\rtl}s. This module provides
a toy code expander for the {\Tt{}Dummy\nwendquote} target. Although it would be
possible to write a generic code expander that works for all targets, we
keep things simple and write a specialized one. The expander is
specified as a set of {\burg} rules.

\item \module{../src/sparcexpander}. This module provides an {\Tt{}expand\nwendquote}
function that re-writes an {\rtl} into a sequence of Sparc machine
instructions. The module is parameterized over a module {\Tt{}I\nwendquote} that
provides constructor functions for abstract Sparc instructions {\Tt{}I.t\nwendquote}
The code expander is implemented as a {\burg} rule set. 

\end{itemize}

% ------------------------------------------------------------------ 
\subsection{Machine Instructions and Recognizer}
% ------------------------------------------------------------------ 

The code expander creates {\rtl}s that represent machine instructions,
which are later matched and emitted by a recognizer. Constructor
functions that build machine instructions and the recognizer must agree
on the representation of individual instructions.

\begin{itemize}
\item \module{../src/minisparc}. Constructor functions and recognizer
for the {\small SPARC}.

\item \module{../gen/sparcmkasm}. Emitter for {\small SPARC}
instructions.

\item Directory {\Tt{}../gen\nwendquote} contains machine-generated constructor
functions and recognizer for the {\small SPARC}. However, they are
currently not used.

\end{itemize}

% ------------------------------------------------------------------ 
\subsection{Utilities}
% ------------------------------------------------------------------ 

The compiler includes a number of self-contained utility modules.

\begin{itemize}
\item Directory {\Tt{}../cllib\nwendquote} contains modules for pretty printing and
parser combinators.

\item \module{../src/topsort}. Topological sort function.
\item \module{../src/idgen}. Generator for unique names.
\item \module{../src/newer}. Tool to compare file dates; used in
      {\Tt{}mkfile\nwendquote}.
\item \module{../src/strutil}. Sets and maps over strings.
\item \module{../src/impossible}. Exception for internal errors.
\item \module{../rtl/uint64}. Unsigned operations on 64 bit intergers.
\item \module{../src/auxfuns}. Various small utilities.
\item \module{../src/base}. Type aliases for {\ocaml} base types. Once
      used by Daniel Damien but unused as of today.
\item \module{../src/mangle}. A name mangler; used by assemblers.
\end{itemize}


\end{document}
\nwfilename{index.nw}
